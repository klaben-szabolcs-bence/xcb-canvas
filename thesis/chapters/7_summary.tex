\Chapter{Összefoglalás}

A dolgozat írása közben nagyon feltűnővé vált, hogy milyen nagy a Canvas API, és hogy mennyire nehéz ilyen alacsony szinten grafikusan programozni. A másik észrevétel, hogy a böngészőben futó kód az cross-platform, amit még nehezebb lenne elérni.

A dolgozat a HTML Canvas API néhány egyszerűbb rajzolási módját részletezte tervezés és implementáció szintjén egyaránt. Sor került a JavaScript-ből elérhető függvények és az XCB adta lehetőségek összevetésére. Ebből jól látszott, hogy közel sincs 1:1 kapcsolat a kétféle API funkcióit illetően.

A dolgozat megírása során elkészült egy XCB alapú keretrendszer, mely az API kialakításában hasonló, mint a HTML5 Canvas, továbbá natív kódra fordítható.

A Canvas API-nak a kb. a 3/8-a van lefedve, így bőven lenne még mit implementálni, kezdve a görbék rajzolásától, saját betűtípus használhatóságán, áttetszőség, színkeverés, és színátmeneteken keresztül, animációk és felhasználói bementtel bezárva. Ezen kívül a cross-platform megoldás és a fejlesztett program gyorsabb ellenőrzésének -- például "hot-swap"-pal -- elérhetővé tétele. További tesztelésre (például egységtesztek írására) szintén szükség lenne.

A minimális siker ellenére, sikernek sikert ért el a program abban, hogy kisebb erőforrásigényű legyen.
