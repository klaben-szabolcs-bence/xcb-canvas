\Chapter{Tervezés}

A fejezet részletesen bemutatja a HTML5 Canvas API C-s implementációjának tervezési folyamatát. Ebben bemutatásra kerülnek az objektum orientált szemlélet szerint felbontott elemek, melyeknek C nyelv esetében a struktúrák és a hozzájuk tartozó függvények felelnek meg.

\Section{Áttekintés}

A keretrendszernek képesnek kell lennie
\begin{itemize}
    \item a képernyőre rajzolni,
    \item egy programozói felületet adni, amivel egyszerű rajzolni és
    \item ehhez szükséges adatokat és állapotokat tárolni.
\end{itemize}
Erre a három feladatra jutott egy-egy osztály.
Ezt \aref{fig:overview-uml}. ábra szemlélteti.

\begin{figure}[h!]
\centering
\begin{tikzpicture}
    \begin{umlpackage}{p-xcb-canvas}
        \umlemptyclass[x=0,y=0]{canvas}
        \umlemptyclass[x=0,y=-2]{canvas-rendering-context}
        \umlemptyclass[x=-2,y=-5]{path-t}
        \umlemptyclass[x=2,y=-5]{xcbcanvas-font-t}
        \umlemptyclass[x=0,y=-7]{xcb-canvas}
    \end{umlpackage}
    \umlemptypackage[y=-10]{xcb}
    \umlVHimport[anchors=-125 and west,arm1=-1cm,arm2=-1cm]{p-xcb-canvas}{xcb}
    \umlunicompo[anchors=-157 and north]{canvas-rendering-context}{path-t}
    \umlunicompo[anchors=-23 and north]{canvas-rendering-context}{xcbcanvas-font-t}
    \umluniassoc[anchors=-110 and 110]{canvas-rendering-context}{xcb-canvas}
    \umldep{canvas}{canvas-rendering-context}
\end{tikzpicture}
\caption{Áttekintő UML az osztályokról}
\label{fig:overview-uml}
\end{figure}

\Section{Képernyőre rajzolás}

Az XCB keretrendszerrel való kommunikációt absztrahálja, és az ahhoz szükséges globális változókat tárolja. Ez pontosabban az X-szerverrel való kapcsolatot, az ablakunk ID-jét és a X-nek fontos grafikai állapotokat tartalmazó struktúrát jelenti. Kerültek továbbá ide függvények, ami XCB segítségével rajzol a képernyőre valamilyen primitívet (téglalap vagy ív), vagy az ablak egy tulajdonságát (például mérete vagy címe) módosítja. Emellett ide jutott az ablakot létrehozó kód is.
Ezt \aref{fig:xcb-uml}. ábra szemlélteti.

\Aref{fig:xcb-uml}. ábrán a plusz- és mínuszjel a láthatóságot jelzi, amely lehet publikus vagy privát. C-ben úgy kerül megvalósításra, hogy csak a forráskódban -- a header fájlban nem -- kerül definiálásra a struktúra \cite{c-oop}.

\begin{figure}[h!]
\centering
\begin{tikzpicture}
    \begin{umlpackage}{p-xcb-canvas}
        \umlclass[]{xcb-canvas}{
        - connection : xcb\_connection \\ - screen : xcb\_screen \\ - window : xcb\_window \\ - gc : xcb\_gcontext
        }
        {
        xcbcanvas\_init\_xcb(canvas : xcbcanvas) : void \\ xcbcanvas\_handle\_events(\\\ ctx : canvas\_rendering\_context,\\\  canvas : xcbcanvas,\\\ draw\_function : void(canvas\_rendering\_context)) \\ xcbcanvas\_load\_font(canvas : xcbcanvas, font\_name : string) \\ xcbcanvas\_set\_window\_title(canvas : xcbcanvas, title : string) \\ xcbcanvas\_set\_color(canvas : xcbcanvas, color : int) \\ xcbcanvas\_set\_stroke\_width(canvas : xcbcanvas, width : uint) \\ xcbcanvas\_stroke\_rectangle(canvas : xcbcanvas, x : int, y : int,\\width : uint, height : uint) \\ xcbcanvas\_draw\_text(canvas : xcbcanvas, x : int, y : int, text : string) \\ xcbcanvas\_line(canvas : xcbcanvas, x1 : int, y1 : int, x2 : int, y2 : int) \\ xcbcanvas\_arc(\\\ canvas : xcbcanvas, \\\ x : int, y : int, \\\ width : uint, height : uint, \\\ angle1 : int, angle2 : int) \\ xcbcanvas\_fill\_arc(\\\ canvas : xcbcanvas, \\\ x : int, y : int, \\\ width : uint, height : uint, \\\ angle1 : int, angle2 : int) \\ xcbcanvas\_draw\_path(canvas : xcbcanvas, path : path)
        }
    \end{umlpackage}
\end{tikzpicture}
\caption{XCB kezelő osztály UML ábrája}
\label{fig:xcb-uml}
\end{figure}

\pagebreak

\Section{Programozói felület}

A felhasználó által használható metódusok és függvények ebben az osztályban kaptak helyet. Ezek a függvények követik a Canvas API szignatúráját azzal a különbséggel, hogy az ablakra rajzoláshoz extra információt tartalmazó osztályt is átadjuk a függvényeknek.

Előfordul, hogy egy névvel több szignatúrájú függvény van definiálva JS-ben, bár ez nem lenne probléma C++-ban, C-ben az. A probléma egyszerűen orvosolható a függvény nevének a módosításával, például az argumentumok számának változtatásával. Az OpenGL hasonló megoldást ad a problémára. 

A \texttt{canvas} struktúrához tartozó függvényeket \aref{fig:canvas-uml}. ábra szemlélteti.

\begin{figure}[h!]
    \centering
    \begin{tikzpicture}
    \begin{umlpackage}{p-xcb-canvas}
        \umlclass[]{canvas}{}{
        canvas\_dealloc(rendering\_context) \\
        canvas\_set\_draw\_function(rendering\_context, draw\_function) \\
        canvas\_stroke\_rectangle(rendering\_context, x, y, width, height) \\
        canvas\_fill\_rectangle(rendering\_context, x, y, width, height) \\
        canvas\_clear\_rectangle(rendering\_context, x, y, width, height) \\
        canvas\_fill\_text(rendering\_context, text, x, y) \\
        canvas\_font4(rendering\_context, family, size, weight, italic) \\
        canvas\_font2(rendering\_context, family, size) \\
        canvas\_set\_line\_width(rendering\_context, width) \\
        canvas\_begin\_path(rendering\_context) \\
        canvas\_move\_to(rendering\_context, x, y) \\
        canvas\_line\_to(rendering\_context, x, y) \\
        canvas\_arc(rendering\_context, x, y, radius, start\_angle, end\_angle) \\
        canvas\_close\_path(rendering\_context) \\
        canvas\_stroke(rendering\_context) \\
        canvas\_fill(rendering\_context)
        }
    \end{umlpackage}
    \end{tikzpicture}
    \caption{Felhasználó felőli osztályok UML ábrája}
    \label{fig:canvas-uml}
\end{figure}

\Section{Az állapotok kezelése}

A különböző állapotok kezeléséhez szükséges adatok, metódusok és függvények osztálya a \texttt{canvas\_rendering\_context}.
A különböző állapotok, amelyeket a keretrendszer kezel az a
\begin{itemize}
    \item betűtípus címe, mérete, kövérsége, dőlése, és
    \item \texttt{Path}-hez kiadott utasítások.
\end{itemize}
A betűtípust kétféle függvénnyel lehet módosítani a \texttt{canvas} osztályból, majd a beállított betűtípus kerül felhasználásra szöveg kirajzolásakor.
A \texttt{Path}-be sorként kerülnek az utasítások bele \texttt{sub\_path}-ba, ami dinamikus tömbként van megoldva.
Továbbá tárolásra kerül a tömb mérete, és hogy az alakzat kitöltött-e, vagy csak körvonal.

A \texttt{canvas\_rendering\_context} struktúra elemeit, és a hozzá tartozó függvényeket \aref{fig:data-uml}. ábra szemlélteti.

% TODO Összehasonlitani XCB és Canvas hivasokat

\begin{figure}[h!]
    \centering
    \begin{tikzpicture}
    \begin{umlpackage}{p-xcb-canvas}
        \umlclass{canvas\_rendering\_context}{
        - canvas : xcb-canvas \\ + fill\_color : uint \\ + stroke\_color : uint \\ + path : path \\ + font : xcbcanvas\_font
        }{
        create\_canvas\_rendering\_context(\\\ draw\_function : void()) \\\ : canvas\_rendering\_context \\ ctx\_update\_font(\\\ rendering\_context : canvas\_rendering\_context, \\\ font : xcbcanvas\_font) \\\ : void
        }
        \umlclass[x=-3,y=-7]{xcbcanvas\_font}{
        + family : string \\+ size : uint \\ + weight : string \\ + italic : bool
        }{}
        \umlclass[x=2,y=-7]{path}{
        + sub\_paths : sub\_path[] \\ + count : int \\ + filled : bool
        }{}
    \end{umlpackage}
    \end{tikzpicture}
    \caption{Adatok kezeléséhez szükséges osztályok UML ábrája}
    \label{fig:data-uml}
\end{figure}

\Section{XCB és Canvas hívások összehasonlítása}

Az alábbi \ref{tab:compare}. táblázat összehasonlít néhány függvényt XCB és Canvas között ugyanarra a problémára. A táblázatból hamar feltünik, hogy át kell adni az állapotokat tároló változókat és az XCB-nek struktúraként kell átadni a színt és vastagságot. Valójában itt egy hívással tudnánk ezeket és többet változtatni egyszerre és a változtatások értékeit kell egy tömbbe rakni. Az értékek sorrendje adott. A \texttt{rectangle} struktúrája egy \textit{x}, \textit{y}, \textit{szélesség} és \textit{magasság}ból áll. A \texttt{points} pedig egy pontokból álló tömb, ahol a pont egy \textit{x} és \textit{y}-ból álló struktúra. A szöveg kiírásánál alacsony szintű programozás miatt a szöveg hosszát is át kell adni.

% TODO: Bővíteni, részletezni
\begin{table}[h!]
    \centering
    \caption{Függvények összehasonlítása}
    \label{tab:compare}
    \medskip
    \begin{tabular}{|l|l|l|} \hline
\textbf{Függvény} & \textbf{XCB} & \textbf{Canvas} \\ \hline
\begin{tabular}[c]{@{}l@{}}Szín\\ változtatása\end{tabular} & \begin{tabular}[c]{@{}l@{}}\texttt{xcb\_change\_gc(}\\ \texttt{c, gc,}\\ \texttt{XCB\_GC\_FOREGROUND,}\\ \texttt{\&color});\end{tabular} & \begin{tabular}[c]{@{}l@{}}\texttt{ctx.strokeStyle =}\\ \texttt{"\#FF0000";}\end{tabular} \\ \hline
\begin{tabular}[c]{@{}l@{}}Vonalvastagság\\ változtatása\end{tabular} & \begin{tabular}[c]{@{}l@{}}\texttt{xcb\_change\_gc(}\\ \texttt{c, gc,}\\ \texttt{XCB\_GC\_LINE\_WIDTH,}\\ \texttt{\&width);}\end{tabular} & \begin{tabular}[c]{@{}l@{}}\texttt{ctx.lineWidth =}\\ \texttt{20;}\end{tabular} \\ \hline
\begin{tabular}[c]{@{}l@{}}Kitöltött\\ téglalap\\ kirajzolása\end{tabular} & \begin{tabular}[c]{@{}l@{}}\texttt{xcb\_poly\_fill\_rectangle(}\\ \texttt{c, canvas-\textgreater{}window, gc, 1,}\\ \texttt{\&rectangle);}\end{tabular} & \begin{tabular}[c]{@{}l@{}}\texttt{ctx.fillRect(0, 0,}\\ \texttt{100, 100);}\end{tabular} \\ \hline
Vonal rajzolása & \begin{tabular}[c]{@{}l@{}}\texttt{xcb\_poly\_line(}\\ \texttt{c,}\\ \texttt{XCB\_COORD\_MODE\_ORIGIN,}\\ \texttt{canvas-\textgreater{}window, gc, 2, points);}\end{tabular} & \begin{tabular}[c]{@{}l@{}}\texttt{ctx.lineTo(300,}\\ \texttt{200);}\end{tabular} \\ \hline
Szöveg kiírása & \begin{tabular}[c]{@{}l@{}}\texttt{xcb\_image\_text\_8\_checked(}\\\texttt{canvas-\textgreater{}connection, strlen(text),}\\\texttt{canvas-\textgreater{}window,canvas-\textgreater{}gc,}\\\texttt{x,y,text);}\end{tabular} & \begin{tabular}[c]{@{}l@{}}\texttt{ctx.fillText(}\\\texttt{"Hello World",}\\\texttt{500, 200);}\end{tabular} \\ \hline
\end{tabular}
\end{table}