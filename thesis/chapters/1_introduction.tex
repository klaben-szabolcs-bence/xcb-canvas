\Chapter{Bevezetés}

 A szakdolgozatom egy már létező egyszerűen használható keretrendszer átültetésével foglalkozik. A cél egy olyan UNIX rendszereken futtatható keretrendszer, amire könnyen átrakhatók a Canvas API-val készített programok, és ennek segítségével kevesebb erőforrás használatával futtathatók.
 
 Az inspiráció a szakdolgozatra a szakmai gyakorlat alatt végzett kutatásomból származik, ahol megismerkedtem mélyebben az UNIX megjelenítő rendszerekről.
 
 A dolgozat 2. fejezete a böngésző alapú grafikus alkalmazásokat tekinti át. Ennek kapcsán láthatjuk majd, hogy a JavaScript milyen széles körben elterjedt, és így milyen előnyök származhatnak abból, hogy ha a JavaScript-ben készített alkalmazások később natív környezetben is futtathatóak lesznek majd.
 
 A dolgozat tulajdonképpen a HTML5 technológiára és az X szerverhez tartozó alacsonyabb szintű implementációkra támaszkodik. A 3. fejezet ezeket taglalja, bemutatva a UNIX alapú rendszerekben a megjelenítés módját, és röviden áttekinti a fejlesztéshez használt eszközkészletet.
 
 A 4. fejezetben az alkalmazás terveit, az egyes részek osztálydiagramjait láthatjuk. Ezek esetében OOP stílusú tervezés történ, így (még hogy ha technikailag nem is tekinthető helyesnek) helyenként egyszerűbbnek tűnt a logikai egységekre, mint osztályokra hivatkozni.
 
 Az 5. fejezet a megvalósítás részleteire tér ki. Kódpéldákkal illusztrálva láthatjuk majd, hogy az elkészült C forráskód hogyan épül fel, hogyan működik, milyen technikai megoldásra volt szükség a HTML Canvas-hez hasonló funkcionalitás eléréséhez. A fejezet végén rövid áttekintést kapunk a C fejlesztéshez használt \textit{cmocka} keretrendszerről és annak használatáról.
 
 A 6. fejezet a kapott eredményeket vizsgálja, vagyis hogy az elkészített natív implementáció mennyivel lehet hatékonyabb, mint az, amelyik a böngészőben fut. Ehhez az Edge, Firefox és Chrome web-böngészők kerültek kiválasztásra, mint referenciák. Hisztogramok formájában láthatjuk majd, hogy hogy alakultak a futási idők a demó programokkal végzett mérések során.

 A szakdolgozatom végén található egy használati útmutató, ami bemutatja hogyan lehet a keretrendszert  és példákat futtatni Ubuntu és Windows rendszeren is.